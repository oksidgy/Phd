
{\actuality} 

Автоматический анализ информации из социальных сетей, интернета, сайтов, структурирование как открытых, так и закрытых баз знаний невозможно выполнить качественно без автоматического извлечения содержимого и структуры электронных текстовых документов в данных источниках.

Анализ информации - это извлечение информации такой как текстовая, графическая информация с последующим ее структурированием с целью ее хранения, осуществления дальнейшего поиска по данной информации, вычисления статистистической информации, и ее обобщения. В обработке электронных документов анализ информации сводиться к анализу содержимого документов, а он в свою очередь невозможен без первоначального этапа - извлечения содержимого и структуры из документа.

Под структурой текстового документа понимается (иерархическая структура) иерархическое представление структурных частей документа так, чтобы имелась приоритетность частей документа и возможность навигации по нему.

Наличие информации о содержимом и структуре электронных документов дает возможность структуризации документов. В первую очередь, это необходимо для систем сбора, поиска информации, и последующей интеллектуальной обработки содержимого документов.

Автоматическая обработка электронных текстовых документов является трудной задачей, поскольку документы могут быть представлены в различных форматах, такие как PDF, DOCX, HTML, изображения, а их структура и содержимое сильно варьируется из-за влияния различных предметных областей (например, технические задания, законы, рекламные брошюры или исследовательские работы), для которых данные документы создавались. Поэтому для качественного извлечения информации необходимо учитывать особенности предметной области документа и его формат. К особенностям предметной области документа относят совокупность правил составления документа: правила составления структуры содержимого документа, правила визуального оформления (форматирования) содержимого, тематика текстового содержимого. В качестве предметных областей документов в диссертации будут рассматриваться документы трех разных распространенных типов: “Техническое задание”, “Выпускная квалификационная работа” и “Нормативно-правовой акт”. 

Формат документа - формат файла документа, задающий спецификацию хранения информации (текстовой, графической, табличной) и работы с ней. Для извлечения содержимого текстового документа необходимо уметь обрабатывать его формат согласно спецификации к данному формату.

Среди широкого набора разнообразных форматов электронных документов можно выделить две основные группы. Документы могут быть представлены форматами, такими как PDF с текстовым слоем, HTML, DOCX, и т. д. Такие форматы являются \textbf{структурированными}, то есть в них содержатся структурные теги, позволяющие выделить в документах заголовки разного уровня, списки, таблицы, данные о форматировании. При этом, в каждом формате теги и их виды определены по-разному.

Кроме того, существуют форматы \textbf{неструктурированных} данных, например, изображения или PDF-документ содержащий страницы, являющиеся сканированными копиями напечатанных на бумаге или написанных от руки документов. Такие документы легко воспринимаются человеком, но сложнее всего поддаются автоматическому анализу в силу того, они не содержат ни текстовой (копируемый текст) или структурной (встроенные в формат теги) информации о содержимом документа.

Автоматический анализ информации требует ее представления в структурированном виде. Прежде чем приступать к структурированию документов, необходимо иметь некоторые данные, извлеченные из документа, такие как текст документа и информация о его визуальных свойствах (метаданные форматирования). В случае документов в структурированных форматах (таких как HTML, DOCX) задача извлечения таких данных является несложной и во многих случаях решенной. В случае работы с неструктурированными документами (PDF или изображений), задача автоматической обработки становится сложной и до сих пор, несмотря на острую потребность, является активной областью для многочисленных исследований.

Таким образом, область автоматического извлечения содержимого и структуры документов различных форматов, в частности неструктурированных форматов PDF и изображений остается по сей день вызовом для систем автоматического интеллектуального анализа текстовых электронных документов.


\underline{\textbf{{\objectTXT}}} исследования являются текстовые электронные документы различных предметных областей технические задания, нормативно-правовые акты и Выпускные квалификационные работы, представленные в виде структурированных неструктурированных форматов документов.


\underline{\textbf{{\themeTXT}}} исследования выступают методы автоматического извлечения содержимого и структуры из исследуемых текстовых документов.


{\aim} диссертационной работы является разработка методов и программных средств для автоматического извлечения содержимого и структуры из электронных текстовых документов. Разрабатываемые методы и программные средства должны удовлетворять следующим требованиям:
\begin{itemize}
        \item возможностью расширения для извлечения содержимого для новых форматов документов;
        \item возможностью расширения для извлечения структуры для новых предметных областей документов, в данной работе рассматриваются текстовые документы технических заданий, нормативно-правовых актов, выпускных квалифицированных работ;
        \item автоматической обработки текстовых электронных документов рассматриваемых форматов и предметных областей.
\end{itemize}


Для достижения поставленной цели необходимо было решить следующие \underline{\textbf{{\tasksTXT}}}:
\begin{itemize}
        \item Разработать метод \textit{автоматического определения корректности} текстового слоя PDF-документов для автоматического выбора метода извлечения содержимого;
        \item Разработать методы автоматического извлечения \textit{содержимого} из форматов документов PDF и изображений сканированных документов;
        \item Разработать метод автоматического извлечения \textit{структуры} из содержимого текстовых документов;
        \item Реализовать предложенные методы в виде программного комплекса, обладающего возможностью расширения новыми форматами и типами текстовых документов.
\end{itemize}


\underline{\textbf{{\noveltyTXT}}} данной диссертационной работы заключается в разработке новых методов и программных средств для автоматического извлечения как содержимого, так и логической структуры из различных форматов текстовых электронных документов:
\begin{enumerate}
    \item Разработан новый метод автоматической проверки корректности текстового слоя PDF-документов, для повышения эффективности их автоматической обработки для русского и английского языков;
    \item Разработан новый метод автоматического извлечения логической структуры из содержимого документов. Метод показывает более высокое качество извлечения логической структуры, по сравнению с другими методами на наборе данных соревнования FINTOC2022;
    \item Разработан расширяемый программный комплекс для автоматического извлечения как содержимого, так и логической структуры из различных форматов и предметных областей текстовых электронных документов в унифицированном виде.
\end{enumerate}


\underline{\textbf{{\influenceTXT}}} По результатам исследований в диссертации разработано новое открытое программное средство для автоматического извлечения содержимого и логической структуры из текстовых электронных документов различных форматов и предметных областей, которое может быть использовано в качестве первоначального этапа для систем автоматической интеллектуальной обработки электронных документов. Программное средство внедрено в открытую библиотеку LangChain\footnote{https://www.langchain.com/}.



\underline{\textbf{{\methodsTXT}}} В данной диссертационной работе применялись методы обработки изображений, машинного обучения, теории вероятностей и оптимизации.


\underline{\textbf{{\defpositionsTXT}}}
\begin{itemize}
        \item В рамках подхода автоматического извлечения содержимого и структуры из электронных текстовых документов, в рамках которого разработаны:
        \begin{itemize}
            \item новый метод автоматической проверки корректности текстового слоя PDF документов для повышения эффективности обработки PDF формата с целью извлечения текстового содержимого документа;
            \item новый метод автоматического извлечения иерархической структуры из содержимого текстовых документов, в том числе для разных предметных областей документов;
        \end{itemize}
        \item Разработан новый программный комплекс в виде библиотеки DEDOC\footnote{https://github.com/ispras/dedoc} для задачи автоматического извлечения содержимого и иерархической структуры из электронных текстовых документов структурированных и неструктурированных форматов в унифицированном виде. Данный комплекс обладает возможностью расширения для поддержки новых форматов и типов текстовых документов.
\end{itemize}



\underline{\textbf{{\probationTXT}}} Результаты данной работы докладывались на конференциях, форумах:
\begin{enumerate}
    \item IVMEM2019 Международная конференция ”Иванниковские чтения 2019”, Великий Новгород, 2019, РФ;
    \item FNP 2021 The 3rd Financial Narrative Processing Workshop, 2021, Marseille, France; LREC;
    \item IVMEM2022 Международная конференция ”Иванниковские чтения 2022”, 2022, Казань, РФ; 
    \item ISPRAS OPEN 2022 Открытая конференция ИСП РАН им. В.П. Иванникова, Москва, РФ;
    \item AINL: Artificial Intelligence and Natural Language Conference, 2023, Ереван, РА;
    \item ISPRAS OPEN 2023 Открытая конференция ИСП РАН им. В.П. Иванникова, 2023, Москва, РФ;
    \item IVMEM2024 Международная конференция ”Иванниковские чтения 2024”, 2024, Великий Новгород, РФ;
    \item DataFest 2024, в гостях у VK, 2024, Москва, РФ;
    \item Гравитация. Международная университетская премия в области искусственного интеллекта и больших данных, 2024, Москва, РФ.
\end{enumerate}



\underline{\textbf{{\publicationsTXT}}} Автор имеет 10 публикаций в печатных изданиях, 8 работ индексируются в Scopus и Web of science. Основные результаты по теме диссертации изложены в 6  печатных изданиях, 6 из которых изданы в журналах, рекомендованных ВАК. Получено 3 свидетельства о регистрации программ для ЭВМ. 

В работах [1-8] автором проведено исследование предметной области, выполнен основной объем теоретических и экспериментальных исследований. В работах [2, 3, 6] Беляевой О.В. и Козлову И. принадлежит постановка задачи, разработка подхода и анализ экспериментов.  Работы [1, 4, 5, 7, 9, 10] выполнены под непосредственным руководством Беляевой О.В. В работе [5] автором разработан подход и метод исправления ориентации, разработка экспериментов и анализ результатов проводилась совместно с соавторами. Работа [7] выполнена полностью автором, редакторские правки и анализ результатов выполнялись совместно соавторами. По теме диссертации имеется 3 свидетельства о государственной регистрации программы для ЭВМ [11-12].


\underline{\textbf{{\contributionTXT}}} Предлагаемые в диссертации инструменты, текстовые наборы данных и исследования разработаны и выполнены автором или при его непосредственном участии.


\underline{\textbf{{\implementationTXT}}} Результаты, полученные в рамках данной работы, внедрены в следующих учереждениях:
\begin{enumerate}
    \item Внедрены непосредственно в систему распознавания первичных документов “NeuroDOC” в учреждении ЗАО “ЕС Лизинг”;
    \item Внедрены непосредственно в платформу Талисман, используемую в Московском государственном институте международных отношений (университет) Министерства иностранных дел Российской Федерации;
\end{enumerate}


% {\progress}
% Этот раздел должен быть отдельным структурным элементом по
% ГОСТ, но он, как правило, включается в описание актуальности
% темы. Нужен он отдельным структурынм элемементом или нет ---
% смотрите другие диссертации вашего совета, скорее всего не нужен.


{\aim} данной работы является \ldots


Для~достижения поставленной цели необходимо было решить следующие {\tasks}:
\begin{enumerate}[beginpenalty=10000] % https://tex.stackexchange.com/a/476052/104425
  \item Исследовать, разработать, вычислить и~т.\:д. и~т.\:п.
  \item Исследовать, разработать, вычислить и~т.\:д. и~т.\:п.
  \item Исследовать, разработать, вычислить и~т.\:д. и~т.\:п.
  \item Исследовать, разработать, вычислить и~т.\:д. и~т.\:п.
\end{enumerate}


{\novelty}
\begin{enumerate}[beginpenalty=10000] % https://tex.stackexchange.com/a/476052/104425
  \item Впервые \ldots
  \item Впервые \ldots
  \item Было выполнено оригинальное исследование \ldots
\end{enumerate}

{\influence} \ldots

{\methods} \ldots

{\defpositions}
\begin{enumerate}[beginpenalty=10000] % https://tex.stackexchange.com/a/476052/104425
  \item Первое положение
  \item Второе положение
  \item Третье положение
  \item Четвертое положение
\end{enumerate}
В папке Documents можно ознакомиться с решением совета из Томского~ГУ
(в~файле \verb+Def_positions.pdf+), где обоснованно даются рекомендации
по~формулировкам защищаемых положений.

{\reliability} полученных результатов обеспечивается \ldots \ Результаты находятся в соответствии с результатами, полученными другими авторами.


{\probation}
Основные результаты работы докладывались~на:
перечисление основных конференций, симпозиумов и~т.\:п.

{\contribution} Автор принимал активное участие \ldots

\ifnumequal{\value{bibliosel}}{0}
{%%% Встроенная реализация с загрузкой файла через движок bibtex8. (При желании, внутри можно использовать обычные ссылки, наподобие `\cite{vakbib1,vakbib2}`).
    {\publications} Основные результаты по теме диссертации изложены
    в~XX~печатных изданиях,
    X из которых изданы в журналах, рекомендованных ВАК,
    X "--- в тезисах докладов.
}%
{%%% Реализация пакетом biblatex через движок biber
    \begin{refsection}[bl-author, bl-registered]
        % Это refsection=1.
        % Процитированные здесь работы:
        %  * подсчитываются, для автоматического составления фразы "Основные результаты ..."
        %  * попадают в авторскую библиографию, при usefootcite==0 и стиле `\insertbiblioauthor` или `\insertbiblioauthorgrouped`
        %  * нумеруются там в зависимости от порядка команд `\printbibliography` в этом разделе.
        %  * при использовании `\insertbiblioauthorgrouped`, порядок команд `\printbibliography` в нём должен быть тем же (см. biblio/biblatex.tex)
        %
        % Невидимый библиографический список для подсчёта количества публикаций:
        \printbibliography[heading=nobibheading, section=1, env=countauthorvak,          keyword=biblioauthorvak]%
        \printbibliography[heading=nobibheading, section=1, env=countauthorwos,          keyword=biblioauthorwos]%
        \printbibliography[heading=nobibheading, section=1, env=countauthorscopus,       keyword=biblioauthorscopus]%
        \printbibliography[heading=nobibheading, section=1, env=countauthorconf,         keyword=biblioauthorconf]%
        \printbibliography[heading=nobibheading, section=1, env=countauthorother,        keyword=biblioauthorother]%
        \printbibliography[heading=nobibheading, section=1, env=countregistered,         keyword=biblioregistered]%
        \printbibliography[heading=nobibheading, section=1, env=countauthorpatent,       keyword=biblioauthorpatent]%
        \printbibliography[heading=nobibheading, section=1, env=countauthorprogram,      keyword=biblioauthorprogram]%
        \printbibliography[heading=nobibheading, section=1, env=countauthor,             keyword=biblioauthor]%
        \printbibliography[heading=nobibheading, section=1, env=countauthorvakscopuswos, filter=vakscopuswos]%
        \printbibliography[heading=nobibheading, section=1, env=countauthorscopuswos,    filter=scopuswos]%
        %
        \nocite{*}%
        %
        {\publications} Основные результаты по теме диссертации изложены в~\arabic{citeauthor}~печатных изданиях,
        \arabic{citeauthorvak} из которых изданы в журналах, рекомендованных ВАК\sloppy%
        \ifnum \value{citeauthorscopuswos}>0%
            , \arabic{citeauthorscopuswos} "--- в~периодических научных журналах, индексируемых Web of~Science и Scopus\sloppy%
        \fi%
        \ifnum \value{citeauthorconf}>0%
            , \arabic{citeauthorconf} "--- в~тезисах докладов.
        \else%
            .
        \fi%
        \ifnum \value{citeregistered}=1%
            \ifnum \value{citeauthorpatent}=1%
                Зарегистрирован \arabic{citeauthorpatent} патент.
            \fi%
            \ifnum \value{citeauthorprogram}=1%
                Зарегистрирована \arabic{citeauthorprogram} программа для ЭВМ.
            \fi%
        \fi%
        \ifnum \value{citeregistered}>1%
            Зарегистрированы\ %
            \ifnum \value{citeauthorpatent}>0%
            \formbytotal{citeauthorpatent}{патент}{}{а}{}\sloppy%
            \ifnum \value{citeauthorprogram}=0 . \else \ и~\fi%
            \fi%
            \ifnum \value{citeauthorprogram}>0%
            \formbytotal{citeauthorprogram}{программ}{а}{ы}{} для ЭВМ.
            \fi%
        \fi%
        % К публикациям, в которых излагаются основные научные результаты диссертации на соискание учёной
        % степени, в рецензируемых изданиях приравниваются патенты на изобретения, патенты (свидетельства) на
        % полезную модель, патенты на промышленный образец, патенты на селекционные достижения, свидетельства
        % на программу для электронных вычислительных машин, базу данных, топологию интегральных микросхем,
        % зарегистрированные в установленном порядке.(в ред. Постановления Правительства РФ от 21.04.2016 N 335)
    \end{refsection}%
    \begin{refsection}[bl-author, bl-registered]
        % Это refsection=2.
        % Процитированные здесь работы:
        %  * попадают в авторскую библиографию, при usefootcite==0 и стиле `\insertbiblioauthorimportant`.
        %  * ни на что не влияют в противном случае
        \nocite{vakbib2}%vak
        \nocite{patbib1}%patent
        \nocite{progbib1}%program
        \nocite{bib1}%other
        \nocite{confbib1}%conf
    \end{refsection}%
        %
        % Всё, что вне этих двух refsection, это refsection=0,
        %  * для диссертации - это нормальные ссылки, попадающие в обычную библиографию
        %  * для автореферата:
        %     * при usefootcite==0, ссылка корректно сработает только для источника из `external.bib`. Для своих работ --- напечатает "[0]" (и даже Warning не вылезет).
        %     * при usefootcite==1, ссылка сработает нормально. В авторской библиографии будут только процитированные в refsection=0 работы.
}

При использовании пакета \verb!biblatex! будут подсчитаны все работы, добавленные
в файл \verb!biblio/author.bib!. Для правильного подсчёта работ в~различных
системах цитирования требуется использовать поля:
\begin{itemize}
        \item \texttt{authorvak} если публикация индексирована ВАК,
        \item \texttt{authorscopus} если публикация индексирована Scopus,
        \item \texttt{authorwos} если публикация индексирована Web of Science,
        \item \texttt{authorconf} для докладов конференций,
        \item \texttt{authorpatent} для патентов,
        \item \texttt{authorprogram} для зарегистрированных программ для ЭВМ,
        \item \texttt{authorother} для других публикаций.
\end{itemize}
Для подсчёта используются счётчики:
\begin{itemize}
        \item \texttt{citeauthorvak} для работ, индексируемых ВАК,
        \item \texttt{citeauthorscopus} для работ, индексируемых Scopus,
        \item \texttt{citeauthorwos} для работ, индексируемых Web of Science,
        \item \texttt{citeauthorvakscopuswos} для работ, индексируемых одной из трёх баз,
        \item \texttt{citeauthorscopuswos} для работ, индексируемых Scopus или Web of~Science,
        \item \texttt{citeauthorconf} для докладов на конференциях,
        \item \texttt{citeauthorother} для остальных работ,
        \item \texttt{citeauthorpatent} для патентов,
        \item \texttt{citeauthorprogram} для зарегистрированных программ для ЭВМ,
        \item \texttt{citeauthor} для суммарного количества работ.
\end{itemize}
% Счётчик \texttt{citeexternal} используется для подсчёта процитированных публикаций;
% \texttt{citeregistered} "--- для подсчёта суммарного количества патентов и программ для ЭВМ.

Для добавления в список публикаций автора работ, которые не были процитированы в
автореферате, требуется их~перечислить с использованием команды \verb!\nocite! в
\verb!Synopsis/content.tex!.
